\documentclass[11pt]{article}
\usepackage[letterpaper, margin=1in]{geometry}
\usepackage{hyperref}
\usepackage{graphicx}
\usepackage{fancyhdr}
\usepackage{xcolor}

% Define colors
\definecolor{headercolor}{RGB}{0, 0, 128}
\definecolor{footerlinecolor}{RGB}{192, 192, 192}

% Header and footer setup
\pagestyle{fancy}
\fancyhf{}
\rfoot{{\color{headercolor} For more information, visit our website:} \url{www.window.state.tx.us/taxinfo/proptax}}
\lfoot{{\color{headercolor} Page 2 - 50-135 - 12-13/12}}
\renewcommand{\headrulewidth}{0pt}
\renewcommand{\footrulewidth}{1pt}
\renewcommand{\footrule}{{\color{footerlinecolor}\hrule width\headwidth height\footrulewidth \vskip-\footrulewidth}}

\begin{document}

% Property Tax Header
\noindent
\textbf{\colorbox{headercolor}{\parbox{\dimexpr\textwidth-2\fboxsep}{\color{white}\textbf{Property Tax \hspace{3mm} Form 50-135}}}}

\vspace{2mm} % Spacing

% Title
\noindent
{\large \textbf{Application for Disabled Veteran’s or Survivor’s Exemption}}

\vspace{2mm} % Spacing

% STEP 2: Property Information
\noindent
\textbf{STEP 2: Property Information}

\noindent
Address, City, State, ZIP Code \underline{\hspace{5.5cm}}

Legal Description (if known) \underline{\hspace{4.5cm}} Appraisal District Account Number (if known) \underline{\hspace{4.5cm}}

Manufactured Home (make, model, and identification number) \underline{\hspace{15cm}}

% STEP 3: Type of Exemption and Qualifications
\noindent
\textbf{STEP 3: Type of Exemption and Qualifications}

\noindent
Check the exemption for which you are applying.

\noindent
\CheckBox[bordercolor = 1 1 1, name=disabledveteran]{} Disabled Veteran’s Exemption

\noindent
\CheckBox[bordercolor = 1 1 1, name=survivingspousechild]{} Surviving Spouse or Child of a Deceased Disabled Veteran

\noindent
\CheckBox[bordercolor = 1 1 1, name=survivingarmedservice]{} Surviving Spouse or Child of Armed Service Member who died on Active Duty

Please provide the following information and attach documentation from the V.A. or service branch identifying the most recent disability rating.

Veteran's Name \underline{\hspace{6cm}}

Branch of Service \underline{\hspace{4.5cm}} Disability Rating \underline{\hspace{4.5cm}} Age \underline{\hspace{2cm}} Serial Number \underline{\hspace{4cm}}

\vspace{2mm} % Spacing

% Further questions with checkboxes
\noindent
Does the service connected disability include:
\begin{itemize}
  \item Loss of one or more limbs \CheckBox[bordercolor = 1 1 1, ]{} Yes \CheckBox[bordercolor = 1 1 1, ]{} No \hspace{10mm} Blindness in one or both eyes \CheckBox[bordercolor = 1 1 1, ]{} Yes \CheckBox[bordercolor = 1 1 1, ]{} No
\end{itemize}

\noindent
Are you the surviving spouse? \hspace{1cm} \CheckBox[bordercolor = 1 1 1, ]{} Yes \CheckBox[bordercolor = 1 1 1, ]{} No

If yes, have you remarried? \hspace{1cm}\CheckBox[bordercolor = 1 1 1, ]{} Yes \CheckBox[bordercolor = 1 1 1, ]{} No

\vspace{2mm} % Spacing

\noindent
Are you a surviving child? \hspace{1.7cm} \CheckBox[bordercolor = 1 1 1, ]{} Yes \CheckBox[bordercolor = 1 1 1, ]{} No

If yes, are you:
\begin{itemize}
  \item Under 18 years of age? \hspace{1.2cm} \CheckBox[bordercolor = 1 1 1, ]{} Yes \CheckBox[bordercolor = 1 1 1, ]{} No
  \item Unmarried? \hspace{2.3cm} \CheckBox[bordercolor = 1 1 1, ]{} Yes \CheckBox[bordercolor = 1 1 1, ]{} No
\end{itemize}

Number of qualifying parent's children who are under 18 and unmarried \underline{\hspace{3cm}}

\vspace{2mm} % Spacing

% STEP 4: Late Application
\noindent
\textbf{STEP 4: Late Application}

\noindent
If you were eligible for this exemption last year, check this box and enter the prior tax year. You must have met all of the qualifications above on January 1 of the prior tax year to receive the exemption for last year.

\noindent
\CheckBox[bordercolor = 1 1 1, name=prioryearexemption]{} Application for exemption for prior tax year, \underline{\hspace{3cm}}.

\vspace{2mm} % Spacing

% STEP 5: Certification and Signature
\noindent
\textbf{STEP 5: Certification and Signature}

By signing this application, you certify the information provided in this application is true and correct to the best of your knowledge and belief.

\vspace{5mm} % Spacing for signature

\noindent
\textbf{Print Name} \underline{\hspace{7cm}} \textbf{Title} \underline{\hspace{7cm}}

\vspace{2mm} % Spacing

\noindent
\textbf{Authorized Signature} \underline{\hspace{7.5cm}} \textbf{Date} \underline{\hspace{3cm}}

\vspace{2mm} % Warning message
\noindent
If you make a false statement on this application, you could be found guilty of a Class A misdemeanor or a state jail felony under Penal Code Section 37.10.

\vspace{2mm} % Spacing

% Footer
\end{document}
